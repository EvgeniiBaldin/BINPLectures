%% -*- coding: utf-8 -*-
\documentclass[12pt,pagesize,paper=192mm:108mm]{scrbook} 
%1920x1080 1280x720
\areaset[current]{192mm}{108mm}
\usepackage{calc}
\usepackage[T2A]{fontenc}
\usepackage[utf8]{inputenc}
\usepackage[english,russian]{babel}
\usepackage{microtype}
\usepackage{misccorr}
\usepackage{cmap}
%\usepackage[unicode=true]{hyperref}
\usepackage{graphicx}
\usepackage{amssymb}
\usepackage{amsmath}
%\usepackage{srcltx}
\usepackage{textcomp}
\usepackage{xspace}
%научные символы и смайлики \smiley \frownie
\usepackage{wasysym}
\usepackage{ccicons}
\begin{document}
\begin{titlepage}
  \vspace*{-1em}
  \begin{center}
    \includegraphics[width=0.23\textwidth]{../NSU-logo}

    Кафедра теоретической физики физического факультета НГУ
    \medskip

    \Large
    Профессор Фадин В.\,С.
    \bigskip

    \huge
    \textbf{Квантовая электродинамика}
    \bigskip

    \Large
    Лекция № 1
    \vfill

    \normalsize
    \vfill

    \normalsize \ccbysa\hspace{0.5em}  Новосибирск 2013
  \end{center}
\end{titlepage}
\newpage

\vspace*{-1em}
\begin{center}
\vfill
  \begin{minipage}{0.65\linewidth}
    Квантовая электродинамика (КЭД) "--- первая успешная квантовая
    теория поля. Лагранжиан свободного дираковского поля. Принцип
    калибровочной инвариантности. Фазовые (калибровочные)
    преобразования первого и второго рода. Электромагнитное поле как
    калибровочное. Абелевы и неабелевы калибровочные теории. Удлинение
    производной.  Лагранжиан калибровочного поля, тензор напряженности
    электромагнитного поля.  Лагранжиан КЭД. Процедура канонического
    квантования. Проблема с каноническим импульсом, сопряженным
    временной компоненте потенциала электромагнитного поля.
    Сохранение лоренц"=инвариантности лагранжиана при каноническом
    квантовании: ввод члена фиксирующего калибровку. Подпространство
    физических состояний.
  \end{minipage}
  \vfill

  \normalsize \ccbysa\hspace{0.5em} Новосибирск 2013
\end{center}
\end{document}
